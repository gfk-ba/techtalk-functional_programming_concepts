\begin{frame}
	\frametitle{Functors a.k.a. First-class Functions}
	\begin{block}{}<+->
		A \alert{functor} is a function object instance that can be referenced by a variable like a regular object, and which can be evaluated by invoking a method on the object.
	\end{block}

	\begin{itemize}[<+-| highlight@+>]
		\item Almost all imperative programming languages support the concept of functor.
		\item There are big differences between languages w.r.t. functors:
		\begin{itemize}[<+-| highlight@+>]
			\item Restrictions that apply to functors;
			\item Support from the language itself;
			\item Support from the standard libraries.
		\end{itemize}
	\end{itemize}
\end{frame}


\begin{frame}
	\frametitle{Using functors to express functional concepts}
	\begin{itemize}[<+-| highlight@+>]
		\item Consider using \alert{pure functions}:
		\begin{itemize}[<+-| highlight@+>]
			\item Side-effect-free.
			\item Context independent.
		\end{itemize}
		\item Next best thing: Side-effect-free functions, but context dependent.
		\item Never use functors with side-effects without a very good reason.
		\begin{itemize}[<+-| highlight@+>]
			\item If you decide to, then try to keep the side-effects as local/close-by as possible.
		\end{itemize}
	\end{itemize}
\end{frame}

