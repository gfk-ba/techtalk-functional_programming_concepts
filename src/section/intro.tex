\begin{frame}
	\frametitle{Goals}
	\begin{itemize}[<+-| highlight@+>]
		\item Re-usable code patterns.
		\item Less code.
		\item Better readable.
		\item Higher performance.
	\end{itemize}
\end{frame}


\begin{frame}
	\frametitle{What is functional programming?}
	\begin{block}{}<+->
		Functional programming is a \alert{declarative programming paradigm} based on pure (i.e. side-effect-free) functions, which are composited and chained to create more complex functions.
	\end{block}

	\begin{block}{}<+->
		Functional programming...
		\begin{itemize}[<+-| highlight@+>]
			\item can make program behavior easier to understand, more predictable and easier to prove correct;
			\item avoids program state and mutable data;
			\item reduces code size; (when using it in an applicable domain);
			\item doesn't tell a computer what to do, but rather how information can be computed from previous information;
			\item encourages modularity;
			\item is inherently optimized for parallel processing.
		\end{itemize}
	\end{block}
\end{frame}


\begin{frame}
	\frametitle{Functional programming: Example}
	\lstinputlisting[language=Erlang]{section/examples/fib.erl}
\end{frame}


\begin{frame}
	\frametitle{Imperative languages}
	\begin{block}{}<+->
		Imperative programming tells a computer what to do by using commands/statements that \alert{change program state}.
	\end{block}
\end{frame}

